\documentclass[12pt,letterpaper]{article}
\usepackage{graphicx,textcomp}
\usepackage{natbib}
\usepackage{setspace}
\usepackage{fullpage}
\usepackage{color}
\usepackage[reqno]{amsmath}
\usepackage{amsthm}
\usepackage{fancyvrb}
\usepackage{amssymb,enumerate}
\usepackage[all]{xy}
\usepackage{endnotes}
\usepackage{lscape}
\newtheorem{com}{Comment}
\usepackage{float}
\usepackage{hyperref}
\newtheorem{lem} {Lemma}
\newtheorem{prop}{Proposition}
\newtheorem{thm}{Theorem}
\newtheorem{defn}{Definition}
\newtheorem{cor}{Corollary}
\newtheorem{obs}{Observation}
\usepackage[compact]{titlesec}
\usepackage{dcolumn}
\usepackage{tikz}
\usetikzlibrary{arrows}
\usepackage{multirow}
\usepackage{xcolor}
\newcolumntype{.}{D{.}{.}{-1}}
\newcolumntype{d}[1]{D{.}{.}{#1}}
\definecolor{light-gray}{gray}{0.65}
\usepackage{url}
\usepackage{listings}
\usepackage{color}

\definecolor{codegreen}{rgb}{0,0.6,0}
\definecolor{codegray}{rgb}{0.5,0.5,0.5}
\definecolor{codepurple}{rgb}{0.58,0,0.82}
\definecolor{backcolour}{rgb}{0.95,0.95,0.92}

\lstdefinestyle{mystyle}{
	backgroundcolor=\color{backcolour},   
	commentstyle=\color{codegreen},
	keywordstyle=\color{magenta},
	numberstyle=\tiny\color{codegray},
	stringstyle=\color{codepurple},
	basicstyle=\footnotesize,
	breakatwhitespace=false,         
	breaklines=true,                 
	captionpos=b,                    
	keepspaces=true,                 
	numbers=left,                    
	numbersep=5pt,                  
	showspaces=false,                
	showstringspaces=false,
	showtabs=false,                  
	tabsize=2
}
\lstset{style=mystyle}
\newcommand{\Sref}[1]{Section~\ref{#1}}
\newtheorem{hyp}{Hypothesis}

\title{Problem Set 6}
\date{Due: May 6, 2020}
\author{QTM 200: Applied Regression Analysis}

\begin{document}
	\maketitle
	
	\section*{Instructions}
	\begin{itemize}
		\item Please show your work! You may lose points by simply writing in the answer. If the problem requires you to execute commands in \texttt{R}, please include the code you used to get your answers. Please also include the \texttt{.R} file that contains your code. If you are not sure if work needs to be shown for a particular problem, please ask.
		\item Your homework should be submitted electronically on the course GitHub page in \texttt{.pdf} form.
		\item This problem set is due before midnight on Wednesday, May 6, 2020. No late assignments will be accepted.
		\item Total available points for this homework is 100.
	\end{itemize}
	
	\vspace{.5cm}
	\section*{Question 1 (50 points): Biology}
	\vspace{.25cm}
	\noindent Load in the data labelled \texttt{cholesterol.csv} on GitHub, which contains an observational study of 315 observations.
	
	\begin{itemize}
		\item
		Response variable: 
		\begin{itemize}
			\item \texttt{cholCat}: 1 if the individual has high cholesterol; 0 if the individual does not have high cholesterol
		\end{itemize}
		\item
		Explanatory variables: 
		\begin{itemize}
			\item
			\texttt{sex}: 1 Male; 0 Female
			\item
			\texttt{fat}: grams of fat consumed per day
			
		\end{itemize}
		
	\end{itemize}
	
	\newpage
	\noindent Please answer the following questions:
	
	\begin{enumerate}
		\item
		We are interested in predicting the cholesterol category based on sex and fat intake.
		\begin{enumerate}
			\item
			Fit an additive model. Provide the summary output, the global null hypothesis, and $p$-value. Please describe the results and provide a conclusion.
			%\item
			%How many iterations did it take to find the maximum likelihood estimates?
			
		\lstinputlisting[language=R, firstline=3, lastline=5]{PS6Answers.R}
		
		
			\begin{verbatim}
		
		Call:
		glm(formula = cholCat ~ sex + fat, family = binomial(link = "logit"), 
		data = cholesterol)
		
		Deviance Residuals: 
		Min        1Q    Median        3Q       Max  
		-2.89662  -0.73093   0.07127   0.64186   2.23806  
		
		Coefficients:
		Estimate Std. Error z value Pr(>|z|)    
		(Intercept) -4.759162   0.563834  -8.441   <2e-16 ***
		sex          1.356750   0.552130   2.457    0.014 *  
		fat          0.065729   0.007826   8.399   <2e-16 ***
		---
		Signif. codes:  0 ‘***’ 0.001 ‘**’ 0.01 ‘*’ 0.05 ‘.’ 0.1 ‘ ’ 1
		
		(Dispersion parameter for binomial family taken to be 1)
		
		Null deviance: 435.54  on 314  degrees of freedom
		Residual deviance: 279.58  on 312  degrees of freedom
		AIC: 285.58
		
		Number of Fisher Scoring iterations: 5
		\end{verbatim}
	
	The global null hypothesis:
	
	$H_{o}: all slopes = 0$
	
	$H_{a}: at least one \beta_{j} not equal to one$
	
	
	The results show that p the value is < 0.01, we can conclude that at least one predictor is reliable in the model.
	
	
		\end{enumerate}
		
		\item
		If explanatory variables are significant in this model, then
		\begin{enumerate}
			\item
			For women, how does increasing their fat intake by 1 gram per day change their odds on being in the high cholesterol group? (Interpretation of a coefficient)
		
		For women, increasing the fat intake by 1 gram, increases the odds of being in high cholesterol group by a multiplicative factor 1.067, it increases the odd by 6.79\%. 
			\item
			For men, how does increasing their fat intake by 1 gram per day change their odds on being in the high cholesterol group? (Interpretation of a coefficient)
			
	For men, by increasing the their fat intake by 1 gram, increases odds of being in a high cholesterol group by a multiplicative factor of 0.0158, meaning it increases the odds by 1.59\%.
			\item
			What is the estimated probability of a woman with a fat intake of 100 grams per day being in the high cholesterol group? 
			
			The estimated probability of a woman with a fat intake of 100 grams per day being in the high cholesterol group is 14\%. 
			\item	
			Would the answers to 2a and 2b potentially change if we included the interaction term in this model? Why? 
		
			\begin{itemize}
				\item Perform a test to see if including an interaction is appropriate.
			\end{itemize}
			%		\item
			%		If you consider all people who have a given fat intake, how does changing from being a female to a male change the odds on being in the high cholesterol group? (For additive model)
	\lstinputlisting[language=R, firstline=7, lastline=8]{PS6Answers.R}
		Interaction model:
		\begin{verbatim}
		Call:
		glm(formula = cholCat ~ sex * fat, family = binomial(link = "logit"), 
		data = cholesterol)
		
		Deviance Residuals: 
		Min        1Q    Median        3Q       Max  
		-2.86893  -0.72131   0.06984   0.65091   2.22120  
		
		Coefficients:
		Estimate Std. Error z value Pr(>|z|)    
		(Intercept) -4.674853   0.587978  -7.951 1.85e-15 ***
		sex          0.541829   1.924729   0.282    0.778    
		fat          0.064513   0.008187   7.880 3.28e-15 ***
		sex:fat      0.012351   0.028011   0.441    0.659    
		---
		Signif. codes:  0 ‘***’ 0.001 ‘**’ 0.01 ‘*’ 0.05 ‘.’ 0.1 ‘ ’ 1
		
		(Dispersion parameter for binomial family taken to be 1)
		
		Null deviance: 435.54  on 314  degrees of freedom
		Residual deviance: 279.37  on 311  degrees of freedom
		AIC: 287.37
		
		Number of Fisher Scoring iterations: 6
		\end{verbatim}
		
		There is no evidence that including an interactive effect of sex and fat intake is a significant predictor for the odds being in a high cholesterol group. 
		
		\end{enumerate}
	\end{enumerate}
	\newpage
	
	
	\section*{Question 2 (50 points): Political Economy}
	\vspace{.25cm}
	\noindent We are interested in how governments' management of public resources impacts economic prosperity. Our data come from \href{https://www.researchgate.net/profile/Adam_Przeworski/publication/240357392_Classifying_Political_Regimes/links/0deec532194849aefa000000/Classifying-Political-Regimes.pdf}{Alvarez, Cheibub, Limongi, and Przeworski (1996)} and is labelled \texttt{gdpChange.csv} on GitHub. The dataset covers 135 countries observed between 1950 or the year of independence or the first year forwhich data on economic growth are available ("entry year"), and 1990 or the last year for which data on economic growth are available ("exit year"). The unit of analysis is a particular country during a particular year, for a total $>$ 3,500 observations. 
	
	\begin{itemize}
		\item
		Response variable: 
		\begin{itemize}
			\item \texttt{GDPWdiff}: Difference in GDP between year $t$ and $t-1$. Possible categories include: "positive", "negative", or "no change"
		\end{itemize}
		\item
		Explanatory variables: 
		\begin{itemize}
			\item
			\texttt{REG}: 1=Democracy; 0=Non-Democracy
			\item
			\texttt{OIL}: 1=if the average ratio of fuel exports to total exports in 1984-86 exceeded 50\%; 0= otherwise
		\end{itemize}
		
	\end{itemize}
	
	\noindent Please answer the following questions:
	
	\begin{enumerate}
		\item Construct and interpret an unordered multinomial logit with \texttt{GDPWdiff} as the output and "no change" as the reference category, including the estimated cutoff points and coefficients.
		
	\begin{verbatim}
		Coefficients:
		(Intercept)       REG        OIL
		no change  -3.8011902 -1.351703 -7.9240683
		positive    0.7284081  0.389905 -0.2076511
		
		Std. Errors:
		(Intercept)        REG        OIL
		no change  0.27014596 0.75825317 32.9772055
		positive   0.04789662 0.07552484  0.1158094
		
		Residual Deviance: 4678.728 
		AIC: 4690.728 
		> 
		\end{verbatim}
	
		Coefficients:
	\begin{verbatim}	
		          (Intercept)       REG          OIL
		no change  0.02234416 0.2587991 0.0003619269
		positive   2.07177984 1.4768404 0.8124904479
		
	\end{verbatim}	
		\item Construct and interpret an ordered multinomial logit with \texttt{GDPWdiff} as the outcome variable, including the estimated cutoff points and coefficients.

		
	\end{enumerate}
	
	
\end{document}